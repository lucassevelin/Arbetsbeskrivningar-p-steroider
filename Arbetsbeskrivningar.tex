\documentclass{article}
\usepackage[swedish]{babel}
\usepackage{graphicx}
\usepackage{datetime}
\usepackage{fontspec}

\usepackage{fancyhdr}
\usepackage{lastpage}
\usepackage{hyperref}
\usepackage{titlesec}
\usepackage{titling}
\usepackage{dirtytalk} % Ger snygga citat-tecken via kommandot \say{}

\usepackage{geometry}
 \geometry
 {
     a4paper,
     total={170mm,257mm},
     top=20mm,
     bottom=30mm,
     left=32.5mm,
     right=32.5mm,
 }

%
% Sätter Huvudfont. Font som är väldigt väldigt lik Garamond
% Garamond finns inte till LaTeX
%
\setmainfont{Junicode}

%
%Sätter fonten på rubrikerna till DINRegular. Funkar inte att ha kursiva rubriker.
%
\newfontfamily\headingfont[]{DINRegular.ttf}
\titleformat*{\section}{\LARGE\headingfont}
\titleformat*{\subsection}{\Large\headingfont}
\titleformat*{\subsubsection}{\large\headingfont}
\renewcommand{\maketitlehooka}{\headingfont}

%
% Deklarerar DIN-fonterna. Används genom att t.ex. skriva {\DINBold(Texten du vill skriva...)}
%
\newfontfamily{\DINBold}{DINBold.ttf}
\newfontfamily{\DINLight}{DINLight.ttf}
\newfontfamily{\DINRegular}{DINRegular.ttf}


%
% Här bestäms formatet på datum.
% 
\newdateformat{mydate}{\THEYEAR-\twodigit{\THEMONTH}-\twodigit{\THEDAY}}
 
%
% Här sätts sidhuvud och sidfot.
%
\pagestyle{fancy}
\fancyhf{} % sets both header and footer to nothing
\renewcommand{\headrulewidth}{0pt}
\rhead{\mydate\today \\ \thepage \hspace{1 pt} (\pageref{LastPage})}
\lhead{\includegraphics[]{lintek.png}}
\cfoot{\footnotesize \textbf{Postadress} LinTek, Tekniska högskolan, 581 83 Linköping \textbf{Besöksadress} Tekniska högskolan, Kårallen, plan 3 \\ \textbf{Telefon} 070-269 45 84 \textbf{Org.nr} 822001-0683 \textbf{Plusgiro} 79 95 13-7 \textbf{Bankgiro} 515-1493 \\
\textbf{E-post} lintek@lintek.liu.se \textbf{Hemsida} \href{www.lintek.liu.se}{www.lintek.liu.se} }

%
% Tar bort indentation
%
\setlength{\parindent}{0cm}
\newcommand{\sectionbreak}{\clearpage}

\begin{document}
%
% Titelsidan
%
\vspace*{9 cm}
\noindent
{\huge\DINBold{Arbetsbeskrivning}}\\

\vspace{0.2cm}
\noindent
Utkast färdigställt den 26 maj 2014 av Niclas Söör, Magdalena Smeds och
Martin Gollvik. Fastställt enligt beslut per capsulam och lagt till
handlingarna på kårstyrelsemöte {[}nr 11, 13/14{]} \\ \\
Revideringar har gjorts:\\ \\
  den 10 november 2014 av Sebastian Brandtberg. Fastställt på
  kårstyrelsemöte {[}nr 5, 14/15{]} \\
  den 11 februari 2015 av Arvid Söderström och Albin Mannerfelt.
  Fastställt på kårstyrelsemöte {[}nr 7, 14/15{]} \\
  den 5 oktober 2015 av Karin Jerner. Fastställt på kårstyrelsemöte
  {[}nr 3, 15/16{]} \\
  den 3 oktober 2016 av Seth Ramström. Fastställt på kårstyrelsemöte
  {[}nr 3, 16/17{]} \\
  den 30 augusti 2017 av Amanda Granqvist. Fastställt på kårstyrelsemöte
  {[}nr 2, 17/18{]} \\
  den 28 juni 2019 av Elin Mattsson, Beatrice Partain och Jakob
  Steneteg. Fastställt enligt presidiebeslut och lagt till handlingarna
  på kårstyrelsemöte {[}nr 1, 19/20{]} \\
  den 10 maj 2020 av Sebastian Carlshamre. Fastställt på kårstyrelsemöte
  {[}nr 12, 19/20{]} \\
  den \today \ av Lucas Sevelin. Aldrig fastställt av kårstyrelsen.

%
% Innehållsförteckningen
%
\vspace*{1 mm}
\tableofcontents

\hypertarget{arbetsbeskrivningar}{%
\section{Arbetsbeskrivningar}\label{arbetsbeskrivningar}}

Kårstyrelsen har tagit fram dessa arbetsbeskrivningar i samråd med
kårledningen.

\hypertarget{syfte}{%
\subsection{Syfte}\label{syfte}}

Arbetsbeskrivningarnas syfte är att bidra till kontinuitet i
kårledningens arbete och att skapa en möjlighet att långsiktigt styra
verksamhetens strategiska fokus. Det görs bland annat genom att fördela
ansvar och arbetsområden till poster och betona särskilt viktiga
arbetsuppgifter inom posters ansvar.

\hypertarget{anvuxe4ndningsomruxe5de}{%
\subsection{Användningsområde}\label{anvuxe4ndningsomruxe5de}}

Arbetsbeskrivningen skall * styra kårledningens arbete och LinTeks
verksamhet * vara ett verktyg i överlämningsarbetet * fungera som ett
informativt dokument, till exempel för en intresserad sökande till en
post eller för en ledamot i kårfullmäktige * öka transparensen i
kårledningens arbete.

\hypertarget{uxf6vrigt}{%
\subsection{Övrigt}\label{uxf6vrigt}}

Dokumentet är tänkt att utvecklas, uppdateras, anpassas och förbättras
både på kort och lång sikt.

\hypertarget{ko}{%
\section{KO}\label{ko}}

\hypertarget{allmuxe4nt}{%
\subsection{Allmänt}\label{allmuxe4nt}}

Kårordförande, KO, är huvudansvarig för ledning av LinTek samt
organisering och övervakning av arbetet inom kårledningen. KO är
heltidsarvoderad under perioden juni till och med juni samt ytterligare
två veckor för efterarbete under sommaren. KO skall säkerställa att
LinTeks verksamhet inom dennes arbetsområde utvärderas och utvecklas,
vilket inkluderar planering och genomförande av den postspecifika
överlämningen till dennes efterträdare samt vara behjälplig i arbetet
med att ta fram budget för nästkommande verksamhetsår.

\hypertarget{arbetsgivaransvar}{%
\subsection{Arbetsgivaransvar}\label{arbetsgivaransvar}}

KO står för arbetsgivaransvaret för LinTeks arvoderade. Detta inkluderar
att ansvara för den psykiska och fysiska arbetsmiljön, så att risken för
ohälsa hos de arvoderade minimeras. Arbetsledning KO står, tillsammans
med vice kårordförande, för ledningen av LinTek. Detta inkluderar att
övervaka den dagliga verksamheten samt upprätthålla god kommunikation
inom hela organisationen. De ansvarar för att kårstyrelsens beslut
implementeras i kårledningen, för att organisera, prioritera och följa
upp arbetet i kårledningen samt för överlämning för tillträdande
personer i kårledningen.

\hypertarget{bolag}{%
\subsection{Bolag}\label{bolag}}

KO är LinTeks representant mot LinTeks bolag och har ansvar för att
hålla sig uppdaterad om vad som händer inom bolagen, både vad gäller
verksamhet, ekonomi och långsiktiga mål. Intern representation KO skall,
tillsammans med vice kårordförande, representera LinTek genom att vara
synlig för, och ha kontakt med, de olika delarna av LinTeks organisation
så som kårfullmäktige, kommittéer och utskott. Detta inkluderar även att
leda ordföranderådet.

\hypertarget{extern-representation}{%
\subsection{Extern representation}\label{extern-representation}}

KO skall representera LinTek externt och är LinTeks ansikte utåt och
därmed primär kontaktperson, gentemot media, politiska aktörer,
kommuner, universitetet, sektioner studentföreningar och andra parter.
KO är ansvarig för LinTeks pressmeddelanden.

\hypertarget{vko}{%
\section{vKO}\label{vko}}

\hypertarget{allmuxe4nt-1}{%
\subsection{Allmänt}\label{allmuxe4nt-1}}

Vice kårordförande, vKO, är huvudansvarig för LinTeks ekonomi, avtal och
arkivering. vKO skall även vara kårordförande behjälplig i dennes
arbetsuppgifter och skall vid kårordförandes frånvaro eller på dennes
förordnande inträda i kårordförandes ställe. vKO är heltidsarvoderad
under perioden juni till och med juni samt ytterligare två veckor för
efterarbete under sommaren. vKO skall säkerställa att LinTeks verksamhet
inom dennes arbetsområde utvärderas och utvecklas, vilket inkluderar
planering och genomförande av den postspecifika överlämningen till
dennes efterträdare.

\hypertarget{ekonomi}{%
\subsection{Ekonomi}\label{ekonomi}}

vKO skall ansvara för LinTeks budget, kontinuerlig bokföring och
redovisning. Detta inkluderar att ansvara för LinTeks löpande ekonomiska
rutiner och resultatuppföljning samt framtagning av budget för
nästkommande verksamhetsår. vKO skall även verka som rådgivare för
LinTeks kårledning, samt leda ekonomigruppen och kassörsrådet. vKO är
LinTeks primära kontaktperson i ekonomiska frågor gentemot enskilda
teknologer, kommuner, universitetet, studentföreningar och andra
aktörer.

\hypertarget{arbetsledning}{%
\subsection{Arbetsledning}\label{arbetsledning}}

vKO står, tillsammans med kårordförande, för ledningen av LinTek. Detta
inkluderar att övervaka den dagliga verksamheten samt upprätthålla god
kommunikation inom hela organisationen. De ansvarar för att
kårstyrelsens beslut implementeras i kårledningen, för att organisera,
prioritera och följa upp arbetet i kårledningen samt för överlämning för
tillträdande personer i kårledningen.

\hypertarget{intern-representation}{%
\subsection{Intern representation}\label{intern-representation}}

vKO skall, tillsammans med kårordförande, representera LinTek genom att
vara synlig för, och ha kontakt, med de olika delarna av LinTeks
organisation så som kårfullmäktige, kommittéer och utskott.

\hypertarget{extern-representation-1}{%
\subsection{Extern representation}\label{extern-representation-1}}

vKO skall representera LinTek externt och är LinTeks ansikte utåt och
därmed sekundär kontaktperson, gentemot media, politiska aktörer,
kommuner, universitetet, sektioner, studentföreningar och andra parter.

\hypertarget{sa}{%
\section{SA}\label{sa}}

\hypertarget{allmuxe4nt-2}{%
\subsection{Allmänt}\label{allmuxe4nt-2}}

Studiesocialt ansvariga, SA, är huvudansvariga för LinTeks studiesociala
verksamhet och skall verka för att alla teknologer har en givande
studietid. LinTek har två stycken SA som är heltidsarvoderade under
perioden januari till och med januari, respektive juni till och med
juni. SA skall säkerställa att LinTeks verksamhet inom dessas
arbetsområde utvärderas och utvecklas, vilket inkluderar planering och
genomförande av den postspecifika överlämningen till dessas efterträdare
samt vara behjälplig i arbetet med att ta fram budget för nästkommande
verksamhetsår.

\hypertarget{studiesocial-kontaktperson}{%
\subsection{Studiesocial
kontaktperson}\label{studiesocial-kontaktperson}}

SA är LinTeks primära kontaktpersoner i alla studiesociala frågor
gentemot enskilda teknologer, kommuner, universitetet, festerier,
fadderier, övriga studentföreningar och andra aktörer. SA med
arvoderingstid januari-januari

\hypertarget{mottagningen}{%
\subsection{Mottagningen}\label{mottagningen}}

Studiesocialt ansvarig med mottagningsansvar, SAm, ska stödja och
koordinera de i LinTeks organisation som har operativt ansvar i
mottagningen. Vidare ansvarar SAm för bedömningar i de frågor där
mottagningspolicyn kan anses ha brutits samt LinTeks del i processen att
revidera mottagningspolicyn. Vidare är SAm direkt ansvarig för den
internationella mottagningen för LinTeks räkning. SAm ansvarar för
kontakten med Studenthälsan gällande planering och genomförande av
fadderutbildningen.

\hypertarget{studentliv}{%
\subsection{Studentliv}\label{studentliv}}

SAm har som uppgift att stötta studentlivet vid LiU. Detta involverar
bland annat att hjälpa organisationer med praktiskt stöd samt ge råd vid
arrangemang. SA med arvoderingstid juni-juni

\hypertarget{arbetsmiljuxf6}{%
\subsection{Arbetsmiljö}\label{arbetsmiljuxf6}}

Studiesocialt ansvarig tillika centralt arbetsmiljöombud, SAc, verkar
för kontinuerlig förbättring av teknologernas välmående. Det innefattar
att arbeta med den fysiska, sociala samt organisatoriska arbetsmiljön.
Därtill arbetar SAc med aktiva åtgärder mot diskriminering och kränkande
behandling. Detta inkluderar representation i de övergripande
universitetsorgan som behandlar frågor i dessa områden. SAc är särskild
kontakt- och resursperson för teknologer som stött på problem inom
områdena arbetsmiljö eller lika villkor. Vidare samordnar och handhar
SAc kontakt med de övriga studeranderepresentanter som i huvudsak
behandlar frågor inom arbetsmiljö eller lika villkor. SAc leder även
arbetsmiljörådet.

\hypertarget{eventutskottet}{%
\subsection{Eventutskottet}\label{eventutskottet}}

Eventutskottet utses och leds av SAc och har i uppgift att stärka
gemenskapen bland LinTeks engagerade genom interna evenemang.

\hypertarget{internationellt-ansvar}{%
\subsection{Internationellt ansvar}\label{internationellt-ansvar}}

SAc, tillsammans med UAu, är huvudansvarig för LinTeks arbete med att
tillvarata internationella teknologers intressen. Detta inkluderar att
vara kontaktperson gentemot studentföreningar eller sektioner som
bedriver internationell verksamhet samt mot enskilda internationella
teknologer, i syfte att skapa samhörighet med resten av universitetet
samt om problem uppstått i samband med studentens studietid. SAc,
tillsammans med UAu, leder även det internationella rådet.

\hypertarget{ua}{%
\section{UA}\label{ua}}

\hypertarget{allmuxe4nt-3}{%
\subsection{Allmänt}\label{allmuxe4nt-3}}

Utbildningsansvariga, UA, är huvudansvariga för LinTeks arbete i
utbildningsfrågor. LinTek har två stycken UA som är heltidsarvoderade
under perioden januari till och med januari, respektive juni till och
med juni. UA skall säkerställa att LinTeks verksamhet inom deras
arbetsområde utvärderas och utvecklas, vilket inkluderar planering och
genomförande av den postspecifika överlämningen till deras efterträdare
samt vara behjälpliga i arbetet med att ta fram budget för nästkommande
verksamhetsår.

\hypertarget{utbildningsbevakning}{%
\subsection{Utbildningsbevakning}\label{utbildningsbevakning}}

UA skall aktivt bedriva påverkansarbete för förbättrad utbildning och
högre utbildningskvalitet för samtliga teknologer samt för att all
utbildning vid LiTH skall vara rättssäker. UA skall särskilt behandla
enskilda teknologers förfrågningar om hjälp i utbildningsrelaterade
ärenden. Detta inkluderar representation gentemot universitetet i
övergripande utbildningsrelaterade frågor samt ledning av
utbildningsrådet.

\hypertarget{kontaktperson-inom-utbildning}{%
\subsection{Kontaktperson inom
utbildning}\label{kontaktperson-inom-utbildning}}

UA är LinTeks primära kontaktpersoner i alla utbildningsfrågor gentemot
enskilda teknologer, enskilda doktorander, kommuner, universitetet,
studentföreningar och andra aktörer.

\hypertarget{ua-med-arvoderingstid-juni-juni}{%
\section{UA med arvoderingstid
juni-juni}\label{ua-med-arvoderingstid-juni-juni}}

Studeranderepresentation Utbildningsansvarig med
studeranderepresentantsansvar, UAs skall ansvara för LinTeks
studeranderepresentation gentemot LiU samt sträva efter att ständigt öka
teknologernas inflytande. Detta inkluderar utbildning av, och kontakt
med, samtliga LinTeks studeranderepresentanter.

\hypertarget{linteks-pedagogikpris-gyllene-moroten}{%
\subsection{LinTeks pedagogikpris, Gyllene
moroten}\label{linteks-pedagogikpris-gyllene-moroten}}

UAs är även ansvarig för LinTeks pedagogikpris, Gyllene moroten. Detta
innebär att sätta ihop en jury för att utse en vinnare, se till att
LinTeks pedagogikpris kommuniceras ut till sektionerna samt se till att
utlämnandet av LinTeks pedagogikpris inträffar.

\hypertarget{internationellt-ansvar-1}{%
\subsection{Internationellt ansvar}\label{internationellt-ansvar-1}}

UAs är huvudansvarig för studenter som väljer att studera utomlands, det
vill säga, utresande studenter.

\hypertarget{doktorandansvar}{%
\subsection{Doktorandansvar}\label{doktorandansvar}}

UAs ansvarar för att upprätthålla kontakten med doktorandsektionen
LiUPhD.

\hypertarget{ua-med-arvoderingstid-januari-januari}{%
\section{UA med arvoderingstid
januari-januari}\label{ua-med-arvoderingstid-januari-januari}}

\hypertarget{representation-gentemot-universitetet}{%
\subsection{Representation gentemot
universitetet}\label{representation-gentemot-universitetet}}

Utbildningsansvarig med universitetsgruppsansvar, UAu, skall ansvara för
kontakten med universitetet gällande utbildningsfrågor samt sträva efter
att ständigt öka teknologernas inflytande på universitetet.

\hypertarget{internationellt-ansvar-2}{%
\subsection{Internationellt ansvar}\label{internationellt-ansvar-2}}

UAu, tillsammans med SAc, är huvudansvarig för LinTeks arbete med att
tillvarata internationella teknologers intressen. Detta inkluderar att
vara kontaktperson för internationella teknologer som har stött på
problem under sin studietid samt att underlätta för dem att få
inflytande över sin utbildning. UAu skall vara LinTeks kontaktperson
gentemot de övergripande organ på universitetet som hanterar inresande
teknologer. UAu, tillsammans med SAc, leder även internationella rådet.

\hypertarget{doktorandansvar-1}{%
\subsection{Doktorandansvar}\label{doktorandansvar-1}}

UAu ansvarar för att driva LinTeks doktorandfrågor mot universitetet.

\hypertarget{na}{%
\section{NA}\label{na}}

\hypertarget{allmuxe4nt-4}{%
\subsection{Allmänt}\label{allmuxe4nt-4}}

Näringslivsansvarig, NA, är huvudansvarig för LinTeks arbete med
arbetsmarknads- och näringslivsfrågor samt ansvarar för LinTeks
kontakter med näringslivet. NA är heltidsarvoderad under perioden juni
till och med juni. NA skall säkerställa att LinTeks verksamhet inom
dennes arbetsområde utvärderas och utvecklas, vilket inkluderar
planering och genomförande av den postspecifika överlämningen till
dennes efterträdare samt vara behjälplig i arbetet med att ta fram
budget för nästkommande verksamhetsår.

\hypertarget{nuxe4ringslivsverksamhet}{%
\subsection{Näringslivsverksamhet}\label{nuxe4ringslivsverksamhet}}

NA ansvarar för LinTeks näringslivsverksamhet. Detta inkluderar att
ansvara för LinTek Näringsliv som arrangerar näringslivsevenemang i
Linköping och Norrköping. Syftet med ett näringslivsevenemang skall vara
att främja teknologernas möjligheter att under studietiden anskaffa sig
relevant erfarenhet inför arbetslivet samt förmedla möjligheten till
långsiktiga kontakter mot näringslivet. Ett årligen återkommande
näringslivsevenemang som NA ansvarar för är LinTeks klimatvecka. NA
skall även leda näringslivsgruppen, näringslivsrådet och alumnirådet,
samt vara ett stöd för fadderiernas sponsgrupp.

\hypertarget{lintek-nuxe4ringsliv}{%
\subsection{LinTek Näringsliv}\label{lintek-nuxe4ringsliv}}

Utskottet utses och leds av NA och är en resurs för NA i arbetet med att
arrangera näringslivsevenemang.

\hypertarget{kontaktperson-i-nuxe4ringslivsfruxe5gor}{%
\subsection{Kontaktperson i
näringslivsfrågor}\label{kontaktperson-i-nuxe4ringslivsfruxe5gor}}

NA är LinTeks primära kontaktperson i alla näringslivsfrågor gentemot
företag, kommuner, universitetet, studentföreningar och andra aktörer.
NA är ansvarig för engagemangsstipendiet.

\hypertarget{samarbetsverksamhet}{%
\subsection{Samarbetsverksamhet}\label{samarbetsverksamhet}}

NA ansvarar för LinTeks samarbetsverksamhet. Syftet med samarbetena
skall vara sådana att de medför förmåner för LinTeks medlemmar. Detta
inkluderar att kontinuerligt söka efter samarbeten med olika parter som
kan medföra förmåner till LinTeks medlemmar. Ansvaret innebär även att
underhålla och sköta de avtal som LinTek slutit i samband med dessa
samarbeten samt upprätthålla en uppdaterad lista över de samarbeten
LinTek har och haft.

\hypertarget{mf}{%
\section{MF}\label{mf}}

\hypertarget{allmuxe4nt-5}{%
\subsection{Allmänt}\label{allmuxe4nt-5}}

Marknadsföringsansvarig, MF, är huvudansvarig för LinTeks informations-
och marknadsföringsarbete. MF är heltidsarvoderad under perioden januari
till och med januari. MF skall säkerställa att LinTeks verksamhet inom
dennes arbetsområde utvärderas och utvecklas, vilket inkluderar
planering och genomförande av den postspecifika överlämningen till
dennes efterträdare samt vara behjälplig i arbetet med att ta fram
budget för nästkommande verksamhetsår.

\hypertarget{varumuxe4rkes--och-informationsarbete}{%
\subsection{Varumärkes- och
informationsarbete}\label{varumuxe4rkes--och-informationsarbete}}

MF är huvudansvarig för att, med LinTeks varumärkes- och
informationsstrategi som utgångspunkt, hålla samman LinTeks
marknadsföring och skapa kontinuitet. Detta inkluderar att
kvalitetssäkra, utveckla och internt informera om LinTeks varumärkes-
och informationsstrategi samt leda marknadsföringsgruppen.

\hypertarget{informationsspridning}{%
\subsection{Informationsspridning}\label{informationsspridning}}

MF är huvudansvarig för LinTeks arbete med att koordinera
informationsspridningen till teknologerna vid tekniska fakulteten. Detta
innefattar att ansvara för LinTeks hemsida, att relevant information
finns tillgänglig på engelska samt LinTeks pressmeddelanden och närvaro
i sociala medier. MF skall även leda informationsrådet.

\hypertarget{medlemsrekrytering}{%
\subsection{Medlemsrekrytering}\label{medlemsrekrytering}}

MF är ansvarig för att koordinera LinTeks arbete med medlemsrekrytering.
Marknadsföringsutskottet Marknadsföringsutskottet utses och leds av MF
och är en resurs för MF i arbetet med marknadsföring och
informationsspridning.

\hypertarget{kontaktperson-i-informations--och-marknadsfuxf6ringsfruxe5gor}{%
\subsection{Kontaktperson i informations- och
marknadsföringsfrågor}\label{kontaktperson-i-informations--och-marknadsfuxf6ringsfruxe5gor}}

MF är LinTeks primära kontaktperson i alla frågor som gäller
marknadsföring och informationsspridning gentemot enskilda teknologer,
kommuner, universitetet, studentföreningar och andra aktörer.

\hypertarget{pl}{%
\section{PL}\label{pl}}

\hypertarget{allmuxe4nt-6}{%
\subsection{Allmänt}\label{allmuxe4nt-6}}

Projektledaren för LARM, PL, ansvarar för planering och genomförande av
LinTeks arbetsmarknadsdagar, LARM. Till sin hjälp har PL en kommitté. PL
är heltidsarvoderad under perioden juni till och med juni. PL skall
säkerställa att LinTeks verksamhet inom dennes arbetsområde utvärderas
och utvecklas, vilket inkluderar planering och genomförande av den
postspecifika överlämningen till dennes efterträdare samt vara
behjälplig i arbetet med att ta fram budget för nästkommande
verksamhetsår.

\hypertarget{larm-kommittuxe9n}{%
\subsection{LARM-kommittén}\label{larm-kommittuxe9n}}

LARM-kommittén utses av PL och skall vara en resurs för planering och
genomförande av LARM. PL arbetsleder kommittén vilket inkluderar att
fastslå kommitténs struktur samt leda kommitténs arbete under projektets
gång. PL ansvarar även för att kommitténs arbete utvärderas och
dokumenteras åt framtida kommittéer.

\hypertarget{representation}{%
\subsection{Representation}\label{representation}}

Utöver arbetet med LARM så skall PL även vara näringslivsansvarig
behjälplig i att kultivera LinTeks näringslivskontakter.

\hypertarget{sof-general}{%
\section{SOF-general}\label{sof-general}}

\hypertarget{allmuxe4nt-7}{%
\subsection{Allmänt}\label{allmuxe4nt-7}}

Studentorkesterfestivalens general, SOF-generalen, är huvudansvarig för
LinTeks arbete med att arrangera Studentorkesterfestivalen i Linköping
under år med udda årtal. Till sin hjälp har SOF- generalen en kommitté.
SOF-generalen är heltidsarvoderad under perioden juni till och med juni.
SOF skall säkerställa att LinTeks verksamhet inom dennes arbetsområde
utvärderas och utvecklas, vilket inkluderar planering och genomförande
av den postspecifika överlämningen till dennes efterträdare samt vara
behjälplig i arbetet med att ta fram budget för nästkommande
verksamhetsår.

\hypertarget{sof-kommittuxe9n}{%
\subsection{SOF-kommittén}\label{sof-kommittuxe9n}}

SOF-kommittén utses av SOF-generalen och skall vara en resurs för
planering och genomförande av Studentorkesterfestivalen. SOF-generalen
arbetsleder kommittén vilket inkluderar att fastslå kommitténs struktur
samt leda kommitténs arbete under projektets gång. SOF-generalen
ansvarar även för att kommitténs arbete utvärderas och dokumenteras åt
framtida kommittéer.

\hypertarget{chefred}{%
\section{ChefRed}\label{chefred}}

\hypertarget{allmuxe4nt-8}{%
\subsection{Allmänt}\label{allmuxe4nt-8}}

Chefredaktör för LiTHanian, ChefRed, är ansvarig utgivare av
medlemstidningen LiTHanian och huvudansvarig för LinTeks arbete med
LiTHanian. ChefRed är därmed primär kontaktperson gällande tidningen.
Till sin hjälp har ChefRed en redaktion. ChefRed skall säkerställa att
LinTeks verksamhet inom dennes arbetsområde utvärderas och utvecklas,
vilket inkluderar planering och genomförande av den postspecifika
överlämningen till dennes efterträdare samt vara behjälplig i arbetet
med att ta fram budget för nästkommande verksamhetsår.

\hypertarget{lithanian-redaktionen}{%
\subsection{LiTHanian-redaktionen}\label{lithanian-redaktionen}}

LiTHanian-redaktionen utses av ChefRed och skall vara en resurs i
produktionen av LiTHanian. ChefRed arbetsleder redaktionen vilket
inkluderar att fastslå redaktionens struktur samt leda redaktionens
arbete under årets gång. ChefRed ansvarar även för att redaktionens
arbete utvärderas och dokumenteras åt framtida redaktioner.

\hypertarget{ma}{%
\section{MA}\label{ma}}

\hypertarget{allmuxe4nt-9}{%
\subsection{Allmänt}\label{allmuxe4nt-9}}

Mottagningsansvariga, MA, är huvudansvariga för LinTeks arbete med att
planera och genomföra LinTeks mottagning. LinTek har två stycken MA, en
med ansvar för mottagningen i Linköping och en med ansvar för
mottagningen i Norrköping. MA är heltidsarvoderade i augusti. MA skall
säkerställa att LinTeks verksamhet inom dennes arbetsområde utvärderas
och utvecklas, vilket inkluderar planering och genomförande av den
postspecifika överlämningen till dennes efterträdare samt vara
behjälplig i arbetet med att ta fram budget för nästkommande
verksamhetsår. MA ansvarar även för att, till kårstyrelsen, lägga fram
förslag på varsin projektplan.

\hypertarget{mottagningspolicy}{%
\subsection{Mottagningspolicy}\label{mottagningspolicy}}

MA bistår Studiesocialt ansvarig i processen vid revidering av
mottagningspolicyn samt ansvarar för att LinTeks mottagningsverksamhet
följer denna mottagningspolicy.

\hypertarget{samordning-av-mottagningsarbetet}{%
\subsection{Samordning av
mottagningsarbetet}\label{samordning-av-mottagningsarbetet}}

MA skall stödja fadderier, ansvariga sektioners faddergrupper och
basårsorganisationer med att genomföra mottagningen. Detta inkluderar
att för dessa planera och genomföra relevant utbildning samt att leda
mottagningsråden. MA är ansvariga för utformandet av fadderhandboken. MA
bistår Studiesocialt ansvarig i planering och genomförande av
fadderutbildning.

\hypertarget{kontaktperson-fuxf6r-mottagningen}{%
\subsection{Kontaktperson för
mottagningen}\label{kontaktperson-fuxf6r-mottagningen}}

MA är LinTeks primära kontaktperson i alla frågor som gäller
mottagningen i respektive stad gentemot enskilda teknologer, kommuner,
universitetet, festerier, fadderier, övriga studentföreningar och andra
aktörer.

\hypertarget{mh-general}{%
\section{MH-general}\label{mh-general}}

\hypertarget{allmuxe4nt-10}{%
\subsection{Allmänt}\label{allmuxe4nt-10}}

München Hobens general, MH-generalen, är huvudansvarig för LinTeks
arbete med att arrangera München Hoben under mottagningen. Till sin
hjälp har MH-generalen en kommitté. MH-generalen är heltidsarvoderad i
augusti. MH-generalen skall säkerställa att LinTeks verksamhet inom
dennes arbetsområde utvärderas och utvecklas, vilket inkluderar
planering och genomförande av den postspecifika överlämningen till
dennes efterträdare samt vara behjälplig i arbetet med att ta fram
budget för nästkommande verksamhetsår.

\hypertarget{muxfcnchen-hoben-kommittuxe9n}{%
\subsection{München
Hoben-kommittén}\label{muxfcnchen-hoben-kommittuxe9n}}

München Hoben-kommittén utses av MH-generalen och skall vara en resurs
för planering och genomförande av München Hoben. MH-generalen
arbetsleder kommittén, vilket inkluderar att fastslå kommitténs struktur
samt leda kommitténs arbete under projektets gång. MH-generalen ansvarar
även för att kommitténs arbete utvärderas och dokumenteras åt framtida
kommittéer.

\# Ordförande för Mattehjälpen \#\# Allmänt Ordförande för mattehjälpen
är huvudansvarig för LinTeks arbete med att anordna räknestugor och
Crash Courses. Ordförande för mattehjälpen skall säkerställa att LinTeks
verksamhet inom dennes arbetsområde utvärderas och utvecklas, vilket
inkluderar planering och genomförande av den postspecifika överlämningen
till dennes efterträdare samt vara behjälplig i arbetet med att ta fram
budget för nästkommande verksamhetsår.

\hypertarget{it}{%
\section{IT}\label{it}}

\hypertarget{allmuxe4nt-11}{%
\subsection{Allmänt}\label{allmuxe4nt-11}}

IT-ansvarig, IT, är huvudansvarig för LinTeks IT-hantering och de
arbetsuppgifter som är kopplade till IT. IT skall säkerställa att
LinTeks verksamhet inom dennes arbetsområde utvärderas och utvecklas,
vilket inkluderar planering och genomförande av den postspecifika
överlämningen till dennes efterträdare samt vara behjälplig i arbetet
med att ta fram budget för nästkommande verksamhetsår.

\hypertarget{linteks-hemsida}{%
\subsection{LinTeks hemsida}\label{linteks-hemsida}}

IT är internt ansvarig för LinTeks hemsida. Det innebär att vara primär
kontaktperson inför kårfullmäktige, kårstyrelsen och kårledningen och
att arrangera utbildning varje år som möjliggör för berörda parter att
hantera innehåll på hemsidan. IT är också ansvarig för kontakten med
leverantören av LinTeks hemsida, i vilket det ingår att upprätthålla
kommunikation såväl som leda framtida utvecklingsarbeten.

\hypertarget{domuxe4ner}{%
\subsection{Domäner}\label{domuxe4ner}}

IT ansvarar för att underhålla LinTeks domäner. I detta ingår
huvudansvaret för subdomänerna. Dokumentation är en viktig del i
arbetet.

\hypertarget{it-utrustning}{%
\subsection{IT-utrustning}\label{it-utrustning}}

LinTek är en stor organisation som är i behov av struktur vad gäller
IT-utrustning. I det ingår inköp, underhåll, inventering och överblick
av reservutrustning. Mjukvarulicenser ingår också. Dokumentation är en
viktig del i arbetet.

\hypertarget{kontaktperson-fuxf6r-linteks-it-relaterade-poster}{%
\subsection{Kontaktperson för LinTeks IT-relaterade
poster}\label{kontaktperson-fuxf6r-linteks-it-relaterade-poster}}

IT är kontaktperson för de personer som jobbar med IT inom LinTeks
utskott. Det innefattar att hålla en struktur för utskottsöverskridande
samarbeten som är till gagn för LinTeks IT- hantering. Ovan nämnda
ansvarsområden bör skötas i samråd med LinTeks engagerade inom IT.


\end{document}
